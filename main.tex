\documentclass[fleqn]{book}

\usepackage{mathtools}
\usepackage{makeidx}
\usepackage{xfrac}
\usepackage{amsthm}

\newtheorem{mythm}{Theorem}
\newtheorem{mydef}{Definition}
\newtheorem{mylem}{Lemma}
\newtheorem{mycor}{Corollary}

\usepackage{hyperref}
\hypersetup{
    bookmarks=true,         % show bookmarks bar?
    unicode=false,          % non-Latin characters in Acrobat’s bookmarks
    pdftoolbar=true,        % show Acrobat’s toolbar?
    pdfmenubar=true,        % show Acrobat’s menu?
    pdffitwindow=false,     % window fit to page when opened
    pdfstartview={FitH},    % fits the width of the page to the window
    pdftitle={My title},    % title
    pdfauthor={Author},     % author
    pdfsubject={Subject},   % subject of the document
    pdfcreator={Creator},   % creator of the document
    pdfproducer={Producer}, % producer of the document
    pdfkeywords={keyword1, key2, key3}, % list of keywords
    pdfnewwindow=true,      % links in new PDF window
    colorlinks=true,       % false: boxed links; true: colored links
    linkcolor=blue,          % color of internal links (change box color with linkbordercolor)
    citecolor=green,        % color of links to bibliography
    filecolor=magenta,      % color of file links
    urlcolor=cyan           % color of external links
}
% New definition of square root:
% it defines the new \sqrt in terms of the old one
\usepackage{letltxmacro}
\makeatletter
\let\oldr@@t\r@@t
\def\r@@t#1#2{%
\setbox0=\hbox{$\oldr@@t#1{#2\,}$}\dimen0=\ht0
\advance\dimen0-0.2\ht0
\setbox2=\hbox{\vrule height\ht0 depth -\dimen0}%
{\box0\lower0.4pt\box2}}
\LetLtxMacro{\oldsqrt}{\sqrt}
\renewcommand*{\sqrt}[2][\ ]{\oldsqrt[#1]{#2} }
\makeatother

\usepackage{datetime}
\newdateformat{mydate}{\the\year}

\author{An Open Source Collaboration}
\title{Introduction to Calculus}

\makeindex

\begin{document}

%\maketitle
\input{copyright}

\tableofcontents

% Note:
% \input{filename} imports the commands from filename into the target file;
% it's equivalent to typing all the commands from filename right into the
% current file where the \input line is
%
% \include{filename} essentially does a \clearpage before and after \input{filename},
% together with some magic to switch to another .aux file, and omit the inclusion at
% all if you have an \includeonly without the filename in the argument

% original list of topics
%\chapter{Outline}

\section{Before Calculus}
\begin{enumerate}
\item Graph of a function
\item Linear function
\item Secant line
\item Slope
\item Tangent
\item Concave function
\item Finite difference
\item Radian
\item Factorial
\item Binomial theorem
\item Free variables and bound variables
\end{enumerate}

\section{Limits}
\begin{enumerate}
\item Limits
\item Limit of a function
\item One-sided limit
\item Limit of a sequence
\item Continuity
\item $(\epsilon,\delta)$ - definition of limit
\end{enumerate}

\section{Differential calculus}
\begin{enumerate}
\item Derivative
\item Notation
\item Newton's notation for differentiation
\item Leibniz's notation for differentiation
\item Simplest rules
\item Derivative of a constant
\item Sum rule in differentiation
\item Constant factor rule in differentiation
\item Linearity of differentiation
\item Power rule
\item Derivative (examples)
\item Chain rule
\item local linearization
\item Product rule
\item Quotient rule
\item Inverse functions and differentiation
\item Implicit differentiation
\item Stationary point
\item Maxima and minima
\item First derivative test
\item Second derivative test
\item Extreme value theorem
\item Differential equation
\item Differential operator
\item Newton's method
\item Taylor's theorem
\item Indeterminate form
\item L'H$\hat{\textrm{o}}$pital's rule
\item General Leibniz rule
\item Mean value theorem
\item Logarithmic derivative
\item Differential (calculus)
\item Related rates
\item Regiomontanus' angle maximization problem
\end{enumerate}

\section{Integral Calculus}
\begin{enumerate}
\item Series
\item Riemann Sums
\item Antiderivative/Indefinite integral
\item Simplest rules
\item Sum rule in integration
\item Constant factor rule in integration
\item Linearity of integration
\item Arbitrary constant of integration
\item Fundamental theorem of calculus
\item Integration by parts
\item Inverse chain rule method
\item Integration by substitution
\item Tangent half-angle substitution
\item Differentiation under the integral sign
\item Trigonometric substitution
\item Partial fractions in integration
\item Quadratic integral
\item Proof that $\frac{22}{7}$ exceeds $\pi$
\item Trapezium rule
\item Integral of the secant function
\item Integral of secant cubed
\item Arclength
\end{enumerate}

\section{Special Functions and Numbers}
\begin{enumerate}
\item Natural logarithm
\item e (mathematical constant)
\item Exponential function
\item Hyperbolic angle
\item Hyperbolic function
\item Stirling's approximation
\item Bernoulli numbers
\end{enumerate}

\section{Numerical Integration}
\begin{enumerate}
\item Rectangle method
\item Trapezium rule
\item Simpson's rule
\item Newton-Cotes formulas
\item Gaussian quadrature
\end{enumerate}

\section{Lists and tables}
\begin{enumerate}
\item Table of common limits
\item Table of derivatives
\item Table of integrals
\item Table of mathematical symbols
\item List of integrals
\item List of integrals of rational functions
\item List of integrals of irrational functions
\item List of integrals of trigonometric functions
\item List of integrals of inverse trigonometric functions
\item List of integrals of hyperbolic functions
\item List of integrals of exponential functions
\item List of integrals of logarithmic functions
\item List of integrals of area functions
\end{enumerate}

\section{Multivariable Calculus}
\begin{enumerate}
\item Partial derivative
\item Disk integration
\item Shell integration
\item Gabriel's horn
\item Jacobian matrix
\item Hessian matrix
\item Curvature
\item Green's theorem
\item Divergence theorem
\item Stokes' theorem
\end{enumerate}

\section{Series}
\begin{enumerate}
\item Infinite series
\item Maclaurin series, Taylor series
\item Fourier series
\item Euler–Maclaurin formula
\end{enumerate}




%\chapter{Introduction}

\section{Purpose}
This effort has been undertaken in order to put accessible control of basic calculus text, problem sets, and exam content into the hands of teachers and students.  Our hope is to make this content readily accessible to a wide swath of the general public.

Modify and enjoy!

\section{History}
History is bunk.

\section{Recommended Course-Related}
For an initial, year long high school course, the recommendation is (future ref) to (future ref).
The second year of high school calculus would cover (future ref) to (future ref).

First semester introductory calculus:  (future ref) to (future ref).
Second semester  introductory calculus:  (future ref) to (future ref).


%\chapter{Precalculus Review}

\section{Rules of Algebra}
words

\section{Numbers}
words

\subsection{Rational}
words

\subsection{Irrational}
words


\section{Functions}
What is a function?

\section{Graph of a function}
words

\subsection{Symmetry}
words

\subsection{Reflections}
words

\subsection{Translations, Stretching, and Shrinking}
words


\section{Linear function}
words

\subsection{Secant line}
words

\subsection{Slope}
words

\subsection{Tangent}
words


\section{Concave Functions}
words


\section{Polynomial Functions}
words

\subsection{Graphing Polynomials}
words

\subsection{Multiplying Polynomials - The Binomial Theorem}
words

\subsection{Factoring Polynomials}
words

\subsection{Dividing Polynomials}
words

\subsection{Synthetic Division}
words



\section{Parabolas}
words


\subsection{Algebra of Parabolas}
words

\subsection{Completing the Square}
words


\section{Rational Functions}
words


\subsection{Sign Charts}
words

\subsection{Vertical Asymptotes}
words

\subsection{Horizontal Asymptotes}
words


\section{Inverse Functions}
words

\subsection{Definition of Inverse Function}
words

\subsection{Graph of an Inverse Function}
words

\subsection{Finding an Inverse Function}
words

\section{Exponential Functions}
words

\section{Logarithms}
words

\section{Trigonometric Functions}
words

\subsection{Angles, Degrees, and Radians}
words

\subsection{Graphing Trigonometric Functions}
words

\section{Trigonometric Identities}
words

\section{Solving Trigonometric Equations}
words




\chapter{Limits}
A sequence is an ordered collection of numbers.  Some examples of sequences:
\begin{itemize}
\item $1,1,1,1,1,1,\ldots$ - a boring sequence
\item $1,2,3,4,5,6,\ldots$ - now we are getting somewhere!
\item $1,\frac{1}{2}, \frac{1}{3},\frac{1}{4},\frac{1}{5},\frac{1}{6},\ldots$ - a very famous sequence, more to come.
\item $2,-2,2,-2,2,-2, \ldots$ - the waffler
\end{itemize}
One of the questions commonly asked about sequences is whether or not the sequence is getting close to some value.  For the first sequence, this is not an interesting question.  The number list starts at 1 and stays there.  We are pretty sure that sequence is ``going to 1''.

The second sequence has a clear pattern - it is getting larger with each term.  If someone tries to convince us that this sequence is going to any particular number, say 5 billion, we can look at term 5,000,000,001 to see that this is not the case.  If this sequence goes somewhere numerically, it goes somewhere big.  We call this big numerical place infinity (denoted $\infty$).  Keep in mind, infinity is not a number, it is an idea of something larger than all the numbers.

\section{Limits}
\begin{mythm} Suppose that $f$ and $g$ are functions such that $f(x)=g(x)$ for all $x$ in some open interval containing $a$ except possibly for $a$, then
\[
\lim_{x\rightarrow a} f(x) = \lim_{x\rightarrow a} g(x)
\]
\end{mythm}

\begin{mythm} Suppose that $f$ and $g$ are functions such that the two limits
\[
\lim_{x\rightarrow a} f(x) = f(a) \textrm{ and } \lim_{x\rightarrow a} g(x) = g(a)
\]
exist.  Suppose that $c$ is a constant and suppose that $n$ is a positive integer.  Then
\begin{align}
&\lim_{x\rightarrow a} c = c\\
&\lim_{x\rightarrow a} x = a\\
&\lim_{x\rightarrow a} c f(x) = c  f(a)\\
&\lim_{x\rightarrow a} (f(x)+g(x)) = f(a)+g(a)\\
&\lim_{x\rightarrow a} (f(x)-g(x)) = f(a)-g(a)\\
&\lim_{x\rightarrow a} f(x) g(x) = f(a) g(a)\\
&\lim_{x\rightarrow a} \frac{f(x)}{g(x)} = \frac{f(a)}{g(a)}, \textrm{ provided }g(a) \ne 0\\
&\lim_{x\rightarrow a} x^n = a^n\\
&\lim_{x\rightarrow a} [f(x)]^n = f(a)^n\\
&\lim_{x\rightarrow a} \sqrt[n]{x} = \sqrt[n]{a}\\
&\lim_{x\rightarrow a} \sqrt[n]{f(x)} = \sqrt[n]{f(a)}, \textrm{ provided } f(a) \ge 0 \\
&\textrm{If } f(x) \le g(x) \textrm{ for all } x \ne a \textrm{ then }\lim_{x\rightarrow a} f(x) \le  \lim_{x\rightarrow a} g(x)
\end{align}
\end{mythm}

\begin{mythm} \textbf{Substitution Theorem}
If $f(x)$ is a polynomial or a rational function, then
\[
\lim_{x\rightarrow a} f(x) = f(a)
\]
assuming $f(a)$ is defined.
\end{mythm}


\subsection{Limit of a function}


\subsection{One-sided limit}
words

\section{Limit of a sequence}
words

\section{Continuity}
\begin{mydef} \textbf{Continuity of a function at a point} \index{continuity}
A function  $f$ is said to be continuous at a point  $p$ if
\begin{description}
\item[a)]  $f(x)$ is defined at  $x=p$, and
\item[b)]
\[
\lim_{x\rightarrow p} f(x)=f(p).
\]
\end{description}
In other words, a function  $f$ is continuous at  $p$ if for every  $\epsilon >0$ there exists a  $\delta >0$ such that
\[
|f(x) - f(p)| < \epsilon \textrm{ whenever } |x - p| < \delta
\]
\end{mydef}



\section{Epsilon-Delta Definition of the Limit}
words


\chapter{Differential Calculus}


\section{Derivatives}
words

\subsection{Notation}
words

\subsection{Newton's notation for differentiation}
words

\subsection{Leibniz's notation for differentiation}
words

\section{Derivative Basics}
words

\subsection{Derivative of a constant}
words

\subsection{Sum rule in differentiation}
words

\subsection{Constant factor rule in differentiation}
words

\subsection{Linearity of differentiation}
words

\subsection{Power rule}
words

\section{Local Linearization}
words


\section{Product rule}
words

\section{Quotient rule}
words

\section{Chain rule}
words

\section{Implicit differentiation}
words

\section{Stationary point}
words

\section{Maxima and minima}
words

\section{First derivative test}
words

\section{Second derivative test}
words


\section{Extreme value theorem}
words

\section{Differential equation}
words

\section{Differential operator}
words

\section{Newton's method}
words

\section{Taylor's theorem}
words

\section{Indeterminate form}
words

\section{L'H$\hat{\textrm{o}}$pital's rule}
words

\section{General Leibniz rule}
words

\section{Mean value theorem}
words

\section{Logarithmic derivative}
words

\section{Differential (calculus)}
words

\section{Related rates}
words





\chapter{Integral Calculus}

\section{Introduction to Finite Series}
words


\section{Riemann Sums}
words

\section{Antiderivative/Indefinite integral}
words

\section{Simplest rules}
words

\subsection{Sum rule in integration}
words

\subsection{Constant factor rule in integration}
words

\subsection{Linearity of integration}
words

\subsection{Arbitrary constant of integration}
words


\section{Fundamental theorem of calculus}
words




\section{Introduction to Infinite Series}
words

\subsection{$\textrm{N}^{\textrm{th}}$ Term Test}
words

\subsection{Comparison Test}
words

\subsection{Limit Comparison Test}
words

\subsection{Ratio Test and Root Test}
words

\subsection{Integral Test}
words

\subsection{Absolute and Conditional Convergence}
words



\section{Integration by parts}
words

\section{Inverse chain rule method}
words

\section{Integration by substitution}
words

\section{Tangent half-angle substitution}
words

\section{Differentiation under the integral sign}
words

\section{Trigonometric substitution}
words

\section{Partial fractions in integration}
words

\section{Quadratic integral}
words

\section{Proof that $\frac{22}{7}$ exceeds $\pi$}
words

\section{Trapezium rule}
words

\section{Integral of the secant function}
words

\section{Integral of secant cubed}
words

\section{Arclength}
words


\chapter{Special Functions and Numbers}


\section{Natural logarithm}
words

\section{e (mathematical constant)}
words

\section{Exponential function}
words

\section{Hyperbolic angle}
words

\section{Hyperbolic function}
words

\section{Stirling's approximation}
words

\section{Bernoulli numbers}
words




\chapter{Numerical Integration}


\section{Rectangle method}
words

\section{Trapezium rule}
words

\section{Simpson's rule}
words

\section{Newton-Cotes formulas}
words

\section{Gaussian quadrature}
words






\chapter{Multivariable Calculus}


\section{Partial derivative}
words

\section{Disk integration}
words

\section{Shell integration}
words

\section{Gabriel's horn}
words

\section{Jacobian matrix}
words

\section{Hessian matrix}
words

\section{Curvature}
words

\section{Green's theorem}
words

\section{Divergence theorem}
words

\section{Stokes' theorem}
words




\chapter{Series}


\section{Infinite series}
words

\section{Maclaurin series and Taylor series}
words

\section{Fourier series}
words

\section{Euler–Maclaurin formula}
words





\chapter{Lists and tables}


\section{Table of common limits}
words

\section{Table of derivatives}
words

\section{Table of integrals}
words

\subsection{Integrals of rational functions}
words

\subsection{Integrals of irrational functions}
words

\subsection{Integrals of trigonometric functions}
words

\subsection{Integrals of inverse trigonometric functions}
words

\subsection{Integrals of hyperbolic functions}
words

\subsection{Integrals of exponential functions}
words

\subsection{Integrals of logarithmic functions}
words

\subsection{Integrals of area functions}
words

\section{Table of mathematical symbols}
words







\chapter{Precalculus Useful Formulas}

When I'm alone and life is getting me down, I can always go \ldots $\sfrac{2}{3}$.

\[
    \sqrt{\frac{2n}{\sqrt[3]{x^2+4}}}
\]

\section{Math Symbols}

\begin{eqnarray*}
\textrm{Symbol} & \textrm{Meaning}\\
\forall & \textrm{For All}\\
\in  & \textrm{Contained In}\\
\exists & \textrm{There Exists}
\end{eqnarray*}


\section{Greek Letters}

\begin{tabular}{|c | c | c | c | c |}
\hline
\textbf{Symbol} & \textbf{Pronunciation} & & \textbf{Symbol} & \textbf{Pronunciation}\\
\hline
$\alpha$ & \textrm{alpha} &  & $\beta$ & \textrm{beta}\\
\hline
$\gamma$ and $\Gamma$ & \textrm{gamma and Gamma} &  & $\delta$ and $\Delta$  & \textrm{delta and Delta}\\
\hline
$\epsilon$ and $\varepsilon$ & \textrm{epsilon} &  & $\eta$ & \textrm{eta}\\
\hline
$\theta$ and $\Theta$  & \textrm{theta and Theta} &  & $\iota$ & \textrm{iota}\\
\hline
$\nu$ & \textrm{nu} &  & $\kappa$ & \textrm{kappa}\\
\hline
$\lambda$ and $\Lambda$  & \textrm{lambda and Lambda} &  & $\mu$ & \textrm{mu}\\
\hline
$\zeta$ & \textrm{zeta} &  & $\xi$ and $\Xi$ & \textrm{xi and Xi}\\
\hline
o & \textrm{omicron}  & & $\pi$ and $\Pi$ & \textrm{pi and Pi}\\
\hline
$\rho$ & \textrm{rho} &  & $\sigma$ and $\Sigma$ & \textrm{sigma and Sigma}\\
\hline
$\tau$ & \textrm{tau} &  & $\upsilon$ and $\Upsilon$ & \textrm{upsilon and Upsilon}\\
\hline
$\phi$ and $\Phi$ & \textrm{phi and Phi} &  & $\varphi$ & \textrm{variant phi}\\
\hline
$\chi$ & \textrm{chi} &  & $\psi$ and $\Psi$ & \textrm{psi and Psi}\\
\hline
$\omega$ and $\Omega$& \textrm{omega and Omega} &  & &\\
\hline
\end{tabular}


\input{fdl}

\printindex

\end{document}
