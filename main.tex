\documentclass[fleqn]{book}

\usepackage{mathtools}
\usepackage{makeidx}
\usepackage{xfrac}
\usepackage{amsthm}

\newtheorem{mythm}{Theorem}
\newtheorem{mydef}{Definition}
\newtheorem{mylem}{Lemma}
\newtheorem{mycor}{Corollary}

\usepackage{hyperref}
\hypersetup{
    bookmarks=true,         % show bookmarks bar?
    unicode=false,          % non-Latin characters in Acrobat’s bookmarks
    pdftoolbar=true,        % show Acrobat’s toolbar?
    pdfmenubar=true,        % show Acrobat’s menu?
    pdffitwindow=false,     % window fit to page when opened
    pdfstartview={FitH},    % fits the width of the page to the window
    pdftitle={My title},    % title
    pdfauthor={Author},     % author
    pdfsubject={Subject},   % subject of the document
    pdfcreator={Creator},   % creator of the document
    pdfproducer={Producer}, % producer of the document
    pdfkeywords={keyword1, key2, key3}, % list of keywords
    pdfnewwindow=true,      % links in new PDF window
    colorlinks=true,       % false: boxed links; true: colored links
    linkcolor=blue,          % color of internal links (change box color with linkbordercolor)
    citecolor=green,        % color of links to bibliography
    filecolor=magenta,      % color of file links
    urlcolor=cyan           % color of external links
}
% New definition of square root:
% it defines the new \sqrt in terms of the old one
\usepackage{letltxmacro}
\makeatletter
\let\oldr@@t\r@@t
\def\r@@t#1#2{%
\setbox0=\hbox{$\oldr@@t#1{#2\,}$}\dimen0=\ht0
\advance\dimen0-0.2\ht0
\setbox2=\hbox{\vrule height\ht0 depth -\dimen0}%
{\box0\lower0.4pt\box2}}
\LetLtxMacro{\oldsqrt}{\sqrt}
\renewcommand*{\sqrt}[2][\ ]{\oldsqrt[#1]{#2} }
\makeatother

\usepackage{datetime}
\newdateformat{mydate}{\the\year}

\author{An Open Source Collaboration}
\title{Introduction to Calculus}

\makeindex

\begin{document}

%\maketitle
\copyright \hspace{0.01cm} \mydate\today \hspace{0.01cm}  Shannon Markiewicz.
    Permission is granted to copy, distribute and/or modify this document
    under the terms of the GNU Free Documentation License, Version 1.3
    or any later version published by the Free Software Foundation;
    with no Invariant Sections, no Front-Cover Texts, and no Back-Cover Texts.
    A copy of the license is included in the section entitled ``GNU
    Free Documentation License''.


\tableofcontents

% Note:
% \input{filename} imports the commands from filename into the target file;
% it's equivalent to typing all the commands from filename right into the
% current file where the \input line is
%
% \include{filename} essentially does a \clearpage before and after \input{filename},
% together with some magic to switch to another .aux file, and omit the inclusion at
% all if you have an \includeonly without the filename in the argument

% original list of topics
%\chapter{Outline}

\section{Before Calculus}
\begin{enumerate}
\item Graph of a function
\item Linear function
\item Secant line
\item Slope
\item Tangent
\item Concave function
\item Finite difference
\item Radian
\item Factorial
\item Binomial theorem
\item Free variables and bound variables
\end{enumerate}

\section{Limits}
\begin{enumerate}
\item Limits
\item Limit of a function
\item One-sided limit
\item Limit of a sequence
\item Continuity
\item $(\epsilon,\delta)$ - definition of limit
\end{enumerate}

\section{Differential calculus}
\begin{enumerate}
\item Derivative
\item Notation
\item Newton's notation for differentiation
\item Leibniz's notation for differentiation
\item Simplest rules
\item Derivative of a constant
\item Sum rule in differentiation
\item Constant factor rule in differentiation
\item Linearity of differentiation
\item Power rule
\item Derivative (examples)
\item Chain rule
\item local linearization
\item Product rule
\item Quotient rule
\item Inverse functions and differentiation
\item Implicit differentiation
\item Stationary point
\item Maxima and minima
\item First derivative test
\item Second derivative test
\item Extreme value theorem
\item Differential equation
\item Differential operator
\item Newton's method
\item Taylor's theorem
\item Indeterminate form
\item L'H$\hat{\textrm{o}}$pital's rule
\item General Leibniz rule
\item Mean value theorem
\item Logarithmic derivative
\item Differential (calculus)
\item Related rates
\item Regiomontanus' angle maximization problem
\end{enumerate}

\section{Integral Calculus}
\begin{enumerate}
\item Series
\item Riemann Sums
\item Antiderivative/Indefinite integral
\item Simplest rules
\item Sum rule in integration
\item Constant factor rule in integration
\item Linearity of integration
\item Arbitrary constant of integration
\item Fundamental theorem of calculus
\item Integration by parts
\item Inverse chain rule method
\item Integration by substitution
\item Tangent half-angle substitution
\item Differentiation under the integral sign
\item Trigonometric substitution
\item Partial fractions in integration
\item Quadratic integral
\item Proof that $\frac{22}{7}$ exceeds $\pi$
\item Trapezium rule
\item Integral of the secant function
\item Integral of secant cubed
\item Arclength
\end{enumerate}

\section{Special Functions and Numbers}
\begin{enumerate}
\item Natural logarithm
\item e (mathematical constant)
\item Exponential function
\item Hyperbolic angle
\item Hyperbolic function
\item Stirling's approximation
\item Bernoulli numbers
\end{enumerate}

\section{Numerical Integration}
\begin{enumerate}
\item Rectangle method
\item Trapezium rule
\item Simpson's rule
\item Newton-Cotes formulas
\item Gaussian quadrature
\end{enumerate}

\section{Lists and tables}
\begin{enumerate}
\item Table of common limits
\item Table of derivatives
\item Table of integrals
\item Table of mathematical symbols
\item List of integrals
\item List of integrals of rational functions
\item List of integrals of irrational functions
\item List of integrals of trigonometric functions
\item List of integrals of inverse trigonometric functions
\item List of integrals of hyperbolic functions
\item List of integrals of exponential functions
\item List of integrals of logarithmic functions
\item List of integrals of area functions
\end{enumerate}

\section{Multivariable Calculus}
\begin{enumerate}
\item Partial derivative
\item Disk integration
\item Shell integration
\item Gabriel's horn
\item Jacobian matrix
\item Hessian matrix
\item Curvature
\item Green's theorem
\item Divergence theorem
\item Stokes' theorem
\end{enumerate}

\section{Series}
\begin{enumerate}
\item Infinite series
\item Maclaurin series, Taylor series
\item Fourier series
\item Euler–Maclaurin formula
\end{enumerate}




%\chapter{Introduction}

\section{Purpose}
This effort has been undertaken in order to put accessible control of basic calculus text, problem sets, and exam content into the hands of teachers and students.  Our hope is to make this content readily accessible to a wide swath of the general public.

Modify and enjoy!

\section{History}
History is bunk.

\section{Recommended Course-Related}
For an initial, year long high school course, the recommendation is (future ref) to (future ref).
The second year of high school calculus would cover (future ref) to (future ref).

First semester introductory calculus:  (future ref) to (future ref).
Second semester  introductory calculus:  (future ref) to (future ref).


%\chapter{Precalculus Review}

\section{Rules of Algebra}
words

\section{Numbers}
words

\subsection{Rational}
words

\subsection{Irrational}
words


\section{Functions}
What is a function?

\section{Graph of a function}
words

\subsection{Symmetry}
words

\subsection{Reflections}
words

\subsection{Translations, Stretching, and Shrinking}
words


\section{Linear function}
words

\subsection{Secant line}
words

\subsection{Slope}
words

\subsection{Tangent}
words


\section{Concave Functions}
words


\section{Polynomial Functions}
words

\subsection{Graphing Polynomials}
words

\subsection{Multiplying Polynomials - The Binomial Theorem}
words

\subsection{Factoring Polynomials}
words

\subsection{Dividing Polynomials}
words

\subsection{Synthetic Division}
words



\section{Parabolas}
words


\subsection{Algebra of Parabolas}
words

\subsection{Completing the Square}
words


\section{Rational Functions}
words


\subsection{Sign Charts}
words

\subsection{Vertical Asymptotes}
words

\subsection{Horizontal Asymptotes}
words


\section{Inverse Functions}
words

\subsection{Definition of Inverse Function}
words

\subsection{Graph of an Inverse Function}
words

\subsection{Finding an Inverse Function}
words

\section{Exponential Functions}
words

\section{Logarithms}
words

\section{Trigonometric Functions}
words

\subsection{Angles, Degrees, and Radians}
words

\subsection{Graphing Trigonometric Functions}
words

\section{Trigonometric Identities}
words

\section{Solving Trigonometric Equations}
words




\chapter{Limits}
A sequence is an ordered collection of numbers.  Some examples of sequences:
\begin{itemize}
\item $1,1,1,1,1,1,\ldots$ - a boring sequence
\item $1,2,3,4,5,6,\ldots$ - now we are getting somewhere!
\item $1,\frac{1}{2}, \frac{1}{3},\frac{1}{4},\frac{1}{5},\frac{1}{6},\ldots$ - a very famous sequence, more to come.
\item $2,-2,2,-2,2,-2, \ldots$ - the waffler
\end{itemize}
One of the questions commonly asked about sequences is whether or not the sequence is getting close to some value.  For the first sequence, this is not an interesting question.  The number list starts at 1 and stays there.  We are pretty sure that sequence is ``going to 1''.

The second sequence has a clear pattern - it is getting larger with each term.  If someone tries to convince us that this sequence is going to any particular number, say 5 billion, we can look at term 5,000,000,001 to see that this is not the case.  If this sequence goes somewhere numerically, it goes somewhere big.  We call this big numerical place infinity (denoted $\infty$).  Keep in mind, infinity is not a number, it is an idea of something larger than all the numbers.

\section{Limits}
\begin{mythm} Suppose that $f$ and $g$ are functions such that $f(x)=g(x)$ for all $x$ in some open interval containing $a$ except possibly for $a$, then
\[
\lim_{x\rightarrow a} f(x) = \lim_{x\rightarrow a} g(x)
\]
\end{mythm}

\begin{mythm} Suppose that $f$ and $g$ are functions such that the two limits
\[
\lim_{x\rightarrow a} f(x) = f(a) \textrm{ and } \lim_{x\rightarrow a} g(x) = g(a)
\]
exist.  Suppose that $c$ is a constant and suppose that $n$ is a positive integer.  Then
\begin{align}
&\lim_{x\rightarrow a} c = c\\
&\lim_{x\rightarrow a} x = a\\
&\lim_{x\rightarrow a} c f(x) = c  f(a)\\
&\lim_{x\rightarrow a} (f(x)+g(x)) = f(a)+g(a)\\
&\lim_{x\rightarrow a} (f(x)-g(x)) = f(a)-g(a)\\
&\lim_{x\rightarrow a} f(x) g(x) = f(a) g(a)\\
&\lim_{x\rightarrow a} \frac{f(x)}{g(x)} = \frac{f(a)}{g(a)}, \textrm{ provided }g(a) \ne 0\\
&\lim_{x\rightarrow a} x^n = a^n\\
&\lim_{x\rightarrow a} [f(x)]^n = f(a)^n\\
&\lim_{x\rightarrow a} \sqrt[n]{x} = \sqrt[n]{a}\\
&\lim_{x\rightarrow a} \sqrt[n]{f(x)} = \sqrt[n]{f(a)}, \textrm{ provided } f(a) \ge 0 \\
&\textrm{If } f(x) \le g(x) \textrm{ for all } x \ne a \textrm{ then }\lim_{x\rightarrow a} f(x) \le  \lim_{x\rightarrow a} g(x)
\end{align}
\end{mythm}

\begin{mythm} \textbf{Substitution Theorem}
If $f(x)$ is a polynomial or a rational function, then
\[
\lim_{x\rightarrow a} f(x) = f(a)
\]
assuming $f(a)$ is defined.
\end{mythm}


\subsection{Limit of a function}


\subsection{One-sided limit}
words

\section{Limit of a sequence}
words

\section{Continuity}
\begin{mydef} \textbf{Continuity of a function at a point} \index{continuity}
A function  $f$ is said to be continuous at a point  $p$ if
\begin{description}
\item[a)]  $f(x)$ is defined at  $x=p$, and
\item[b)]
\[
\lim_{x\rightarrow p} f(x)=f(p).
\]
\end{description}
In other words, a function  $f$ is continuous at  $p$ if for every  $\epsilon >0$ there exists a  $\delta >0$ such that
\[
|f(x) - f(p)| < \epsilon \textrm{ whenever } |x - p| < \delta
\]
\end{mydef}



\section{Epsilon-Delta Definition of the Limit}
words


\chapter{Differential Calculus}


\section{Derivatives}
words

\subsection{Notation}
words

\subsection{Newton's notation for differentiation}
words

\subsection{Leibniz's notation for differentiation}
words

\section{Derivative Basics}
words

\subsection{Derivative of a constant}
words

\subsection{Sum rule in differentiation}
words

\subsection{Constant factor rule in differentiation}
words

\subsection{Linearity of differentiation}
words

\subsection{Power rule}
words

\section{Local Linearization}
words


\section{Product rule}
words

\section{Quotient rule}
words

\section{Chain rule}
words

\section{Implicit differentiation}
words

\section{Stationary point}
words

\section{Maxima and minima}
words

\section{First derivative test}
words

\section{Second derivative test}
words


\section{Extreme value theorem}
words

\section{Differential equation}
words

\section{Differential operator}
words

\section{Newton's method}
words

\section{Taylor's theorem}
words

\section{Indeterminate form}
words

\section{L'H$\hat{\textrm{o}}$pital's rule}
words

\section{General Leibniz rule}
words

\section{Mean value theorem}
words

\section{Logarithmic derivative}
words

\section{Differential (calculus)}
words

\section{Related rates}
words





\chapter{Integral Calculus}

\section{Introduction to Finite Series}
words


\section{Riemann Sums}
words

\section{Antiderivative/Indefinite integral}
words

\section{Simplest rules}
words

\subsection{Sum rule in integration}
words

\subsection{Constant factor rule in integration}
words

\subsection{Linearity of integration}
words

\subsection{Arbitrary constant of integration}
words


\section{Fundamental theorem of calculus}
words




\section{Introduction to Infinite Series}
words

\subsection{$\textrm{N}^{\textrm{th}}$ Term Test}
words

\subsection{Comparison Test}
words

\subsection{Limit Comparison Test}
words

\subsection{Ratio Test and Root Test}
words

\subsection{Integral Test}
words

\subsection{Absolute and Conditional Convergence}
words



\section{Integration by parts}
words

\section{Inverse chain rule method}
words

\section{Integration by substitution}
words

\section{Tangent half-angle substitution}
words

\section{Differentiation under the integral sign}
words

\section{Trigonometric substitution}
words

\section{Partial fractions in integration}
words

\section{Quadratic integral}
words

\section{Proof that $\frac{22}{7}$ exceeds $\pi$}
words

\section{Trapezium rule}
words

\section{Integral of the secant function}
words

\section{Integral of secant cubed}
words

\section{Arclength}
words


\chapter{Special Functions and Numbers}


\section{Natural logarithm}
words

\section{e (mathematical constant)}
words

\section{Exponential function}
words

\section{Hyperbolic angle}
words

\section{Hyperbolic function}
words

\section{Stirling's approximation}
words

\section{Bernoulli numbers}
words




\chapter{Numerical Integration}


\section{Rectangle method}
words

\section{Trapezium rule}
words

\section{Simpson's rule}
words

\section{Newton-Cotes formulas}
words

\section{Gaussian quadrature}
words






\chapter{Multivariable Calculus}


\section{Partial derivative}
words

\section{Disk integration}
words

\section{Shell integration}
words

\section{Gabriel's horn}
words

\section{Jacobian matrix}
words

\section{Hessian matrix}
words

\section{Curvature}
words

\section{Green's theorem}
words

\section{Divergence theorem}
words

\section{Stokes' theorem}
words




\chapter{Series}


\section{Infinite series}
words

\section{Maclaurin series and Taylor series}
words

\section{Fourier series}
words

\section{Euler–Maclaurin formula}
words





\chapter{Lists and tables}


\section{Table of common limits}
words

\section{Table of derivatives}
words

\section{Table of integrals}
words

\subsection{Integrals of rational functions}
words

\subsection{Integrals of irrational functions}
words

\subsection{Integrals of trigonometric functions}
words

\subsection{Integrals of inverse trigonometric functions}
words

\subsection{Integrals of hyperbolic functions}
words

\subsection{Integrals of exponential functions}
words

\subsection{Integrals of logarithmic functions}
words

\subsection{Integrals of area functions}
words

\section{Table of mathematical symbols}
words







\chapter{Precalculus Useful Formulas}

When I'm alone and life is getting me down, I can always go \ldots $\sfrac{2}{3}$.

\[
    \sqrt{\frac{2n}{\sqrt[3]{x^2+4}}}
\]

\section{Math Symbols}

\begin{eqnarray*}
\textrm{Symbol} & \textrm{Meaning}\\
\forall & \textrm{For All}\\
\in  & \textrm{Contained In}\\
\exists & \textrm{There Exists}
\end{eqnarray*}


\section{Greek Letters}

\begin{tabular}{|c | c | c | c | c |}
\hline
\textbf{Symbol} & \textbf{Pronunciation} & & \textbf{Symbol} & \textbf{Pronunciation}\\
\hline
$\alpha$ & \textrm{alpha} &  & $\beta$ & \textrm{beta}\\
\hline
$\gamma$ and $\Gamma$ & \textrm{gamma and Gamma} &  & $\delta$ and $\Delta$  & \textrm{delta and Delta}\\
\hline
$\epsilon$ and $\varepsilon$ & \textrm{epsilon} &  & $\eta$ & \textrm{eta}\\
\hline
$\theta$ and $\Theta$  & \textrm{theta and Theta} &  & $\iota$ & \textrm{iota}\\
\hline
$\nu$ & \textrm{nu} &  & $\kappa$ & \textrm{kappa}\\
\hline
$\lambda$ and $\Lambda$  & \textrm{lambda and Lambda} &  & $\mu$ & \textrm{mu}\\
\hline
$\zeta$ & \textrm{zeta} &  & $\xi$ and $\Xi$ & \textrm{xi and Xi}\\
\hline
o & \textrm{omicron}  & & $\pi$ and $\Pi$ & \textrm{pi and Pi}\\
\hline
$\rho$ & \textrm{rho} &  & $\sigma$ and $\Sigma$ & \textrm{sigma and Sigma}\\
\hline
$\tau$ & \textrm{tau} &  & $\upsilon$ and $\Upsilon$ & \textrm{upsilon and Upsilon}\\
\hline
$\phi$ and $\Phi$ & \textrm{phi and Phi} &  & $\varphi$ & \textrm{variant phi}\\
\hline
$\chi$ & \textrm{chi} &  & $\psi$ and $\Psi$ & \textrm{psi and Psi}\\
\hline
$\omega$ and $\Omega$& \textrm{omega and Omega} &  & &\\
\hline
\end{tabular}


\chapter{GNU Free Documentation License}

\begin{center}
Version 1.3, 3 November 2008
\end{center}

Copyright \copyright 2000, 2001, 2002, 2007, 2008 Free Software Foundation, Inc.
\href{http://fsf.org/}
Everyone is permitted to copy and distribute verbatim copies of this license document, but changing it is not allowed.

\begin{enumerate}
\item PREAMBLE

The purpose of this License is to make a manual, textbook, or other
functional and useful document ``free'' in the sense of freedom: to
assure everyone the effective freedom to copy and redistribute it,
with or without modifying it, either commercially or noncommercially.
Secondarily, this License preserves for the author and publisher a way
to get credit for their work, while not being considered responsible
for modifications made by others.

This License is a kind of ``copyleft'', which means that derivative
works of the document must themselves be free in the same sense.  It
complements the GNU General Public License, which is a copyleft
license designed for free software.

We have designed this License in order to use it for manuals for free
software, because free software needs free documentation: a free
program should come with manuals providing the same freedoms that the
software does.  But this License is not limited to software manuals;
it can be used for any textual work, regardless of subject matter or
whether it is published as a printed book.  We recommend this License
principally for works whose purpose is instruction or reference.


\item APPLICABILITY AND DEFINITIONS

This License applies to any manual or other work, in any medium, that
contains a notice placed by the copyright holder saying it can be
distributed under the terms of this License.  Such a notice grants a
world-wide, royalty-free license, unlimited in duration, to use that
work under the conditions stated herein.  The ``Document'', below,
refers to any such manual or work.  Any member of the public is a
licensee, and is addressed as ``you''.  You accept the license if you
copy, modify or distribute the work in a way requiring permission
under copyright law.

A ``Modified Version'' of the Document means any work containing the
Document or a portion of it, either copied verbatim, or with
modifications and/or translated into another language.

A ``Secondary Section'' is a named appendix or a front-matter section of
the Document that deals exclusively with the relationship of the
publishers or authors of the Document to the Document's overall
subject (or to related matters) and contains nothing that could fall
directly within that overall subject.  (Thus, if the Document is in
part a textbook of mathematics, a Secondary Section may not explain
any mathematics.)  The relationship could be a matter of historical
connection with the subject or with related matters, or of legal,
commercial, philosophical, ethical or political position regarding
them.

The ``Invariant Sections'' are certain Secondary Sections whose titles
are designated, as being those of Invariant Sections, in the notice
that says that the Document is released under this License.  If a
section does not fit the above definition of Secondary then it is not
allowed to be designated as Invariant.  The Document may contain zero
Invariant Sections.  If the Document does not identify any Invariant
Sections then there are none.

The ``Cover Texts'' are certain short passages of text that are listed,
as Front-Cover Texts or Back-Cover Texts, in the notice that says that
the Document is released under this License.  A Front-Cover Text may
be at most 5 words, and a Back-Cover Text may be at most 25 words.

A ``Transparent'' copy of the Document means a machine-readable copy,
represented in a format whose specification is available to the
general public, that is suitable for revising the document
straightforwardly with generic text editors or (for images composed of
pixels) generic paint programs or (for drawings) some widely available
drawing editor, and that is suitable for input to text formatters or
for automatic translation to a variety of formats suitable for input
to text formatters.  A copy made in an otherwise Transparent file
format whose markup, or absence of markup, has been arranged to thwart
or discourage subsequent modification by readers is not Transparent.
An image format is not Transparent if used for any substantial amount
of text.  A copy that is not ``Transparent'' is called ``Opaque''.

Examples of suitable formats for Transparent copies include plain
ASCII without markup, Texinfo input format, LaTeX input format, SGML
or XML using a publicly available DTD, and standard-conforming simple
HTML, PostScript or PDF designed for human modification.  Examples of
transparent image formats include PNG, XCF and JPG.  Opaque formats
include proprietary formats that can be read and edited only by
proprietary word processors, SGML or XML for which the DTD and/or
processing tools are not generally available, and the
machine-generated HTML, PostScript or PDF produced by some word
processors for output purposes only.

The ``Title Page'' means, for a printed book, the title page itself,
plus such following pages as are needed to hold, legibly, the material
this License requires to appear in the title page.  For works in
formats which do not have any title page as such, ``Title Page'' means
the text near the most prominent appearance of the work's title,
preceding the beginning of the body of the text.

The ``publisher'' means any person or entity that distributes copies of
the Document to the public.

A section ``Entitled XYZ'' means a named subunit of the Document whose
title either is precisely XYZ or contains XYZ in parentheses following
text that translates XYZ in another language.  (Here XYZ stands for a
specific section name mentioned below, such as ``Acknowledgements'',
``Dedications'', ``Endorsements'', or ``History''.)  To ``Preserve the Title''
of such a section when you modify the Document means that it remains a
section ``Entitled XYZ'' according to this definition.

The Document may include Warranty Disclaimers next to the notice which
states that this License applies to the Document.  These Warranty
Disclaimers are considered to be included by reference in this
License, but only as regards disclaiming warranties: any other
implication that these Warranty Disclaimers may have is void and has
no effect on the meaning of this License.

\item VERBATIM COPYING

You may copy and distribute the Document in any medium, either
commercially or noncommercially, provided that this License, the
copyright notices, and the license notice saying this License applies
to the Document are reproduced in all copies, and that you add no
other conditions whatsoever to those of this License.  You may not use
technical measures to obstruct or control the reading or further
copying of the copies you make or distribute.  However, you may accept
compensation in exchange for copies.  If you distribute a large enough
number of copies you must also follow the conditions in section 3.

You may also lend copies, under the same conditions stated above, and
you may publicly display copies.


\item COPYING IN QUANTITY

If you publish printed copies (or copies in media that commonly have
printed covers) of the Document, numbering more than 100, and the
Document's license notice requires Cover Texts, you must enclose the
copies in covers that carry, clearly and legibly, all these Cover
Texts: Front-Cover Texts on the front cover, and Back-Cover Texts on
the back cover.  Both covers must also clearly and legibly identify
you as the publisher of these copies.  The front cover must present
the full title with all words of the title equally prominent and
visible.  You may add other material on the covers in addition.
Copying with changes limited to the covers, as long as they preserve
the title of the Document and satisfy these conditions, can be treated
as verbatim copying in other respects.

If the required texts for either cover are too voluminous to fit
legibly, you should put the first ones listed (as many as fit
reasonably) on the actual cover, and continue the rest onto adjacent
pages.

If you publish or distribute Opaque copies of the Document numbering
more than 100, you must either include a machine-readable Transparent
copy along with each Opaque copy, or state in or with each Opaque copy
a computer-network location from which the general network-using
public has access to download using public-standard network protocols
a complete Transparent copy of the Document, free of added material.
If you use the latter option, you must take reasonably prudent steps,
when you begin distribution of Opaque copies in quantity, to ensure
that this Transparent copy will remain thus accessible at the stated
location until at least one year after the last time you distribute an
Opaque copy (directly or through your agents or retailers) of that
edition to the public.

It is requested, but not required, that you contact the authors of the
Document well before redistributing any large number of copies, to
give them a chance to provide you with an updated version of the
Document.


\item MODIFICATIONS

You may copy and distribute a Modified Version of the Document under
the conditions of sections 2 and 3 above, provided that you release
the Modified Version under precisely this License, with the Modified
Version filling the role of the Document, thus licensing distribution
and modification of the Modified Version to whoever possesses a copy
of it.  In addition, you must do these things in the Modified Version:
\begin{enumerate}
\item Use in the Title Page (and on the covers, if any) a title distinct
   from that of the Document, and from those of previous versions
   (which should, if there were any, be listed in the History section
   of the Document).  You may use the same title as a previous version
   if the original publisher of that version gives permission.
\item List on the Title Page, as authors, one or more persons or entities
   responsible for authorship of the modifications in the Modified
   Version, together with at least five of the principal authors of the
   Document (all of its principal authors, if it has fewer than five),
   unless they release you from this requirement.
\item State on the Title page the name of the publisher of the
   Modified Version, as the publisher.
\item Preserve all the copyright notices of the Document.
\item Add an appropriate copyright notice for your modifications
   adjacent to the other copyright notices.
\item Include, immediately after the copyright notices, a license notice
   giving the public permission to use the Modified Version under the
   terms of this License, in the form shown in the Addendum below.
\item Preserve in that license notice the full lists of Invariant Sections
   and required Cover Texts given in the Document's license notice.
\item Include an unaltered copy of this License.
\item Preserve the section Entitled ``History'', Preserve its Title, and add
   to it an item stating at least the title, year, new authors, and
   publisher of the Modified Version as given on the Title Page.  If
   there is no section Entitled ``History'' in the Document, create one
   stating the title, year, authors, and publisher of the Document as
   given on its Title Page, then add an item describing the Modified
   Version as stated in the previous sentence.
\item Preserve the network location, if any, given in the Document for
   public access to a Transparent copy of the Document, and likewise
   the network locations given in the Document for previous versions
   it was based on.  These may be placed in the ``History'' section.
   You may omit a network location for a work that was published at
   least four years before the Document itself, or if the original
   publisher of the version it refers to gives permission.
\item For any section Entitled ``Acknowledgements'' or ``Dedications'',
   Preserve the Title of the section, and preserve in the section all
   the substance and tone of each of the contributor acknowledgements
   and/or dedications given therein.
\item Preserve all the Invariant Sections of the Document,
   unaltered in their text and in their titles.  Section numbers
   or the equivalent are not considered part of the section titles.
\item Delete any section Entitled ``Endorsements''.  Such a section
   may not be included in the Modified Version.
\item Do not retitle any existing section to be Entitled ``Endorsements''
   or to conflict in title with any Invariant Section.
\item Preserve any Warranty Disclaimers.
\end{enumerate}

If the Modified Version includes new front-matter sections or
appendices that qualify as Secondary Sections and contain no material
copied from the Document, you may at your option designate some or all
of these sections as invariant.  To do this, add their titles to the
list of Invariant Sections in the Modified Version's license notice.
These titles must be distinct from any other section titles.

You may add a section Entitled ``Endorsements'', provided it contains
nothing but endorsements of your Modified Version by various
parties--for example, statements of peer review or that the text has
been approved by an organization as the authoritative definition of a
standard.

You may add a passage of up to five words as a Front-Cover Text, and a
passage of up to 25 words as a Back-Cover Text, to the end of the list
of Cover Texts in the Modified Version.  Only one passage of
Front-Cover Text and one of Back-Cover Text may be added by (or
through arrangements made by) any one entity.  If the Document already
includes a cover text for the same cover, previously added by you or
by arrangement made by the same entity you are acting on behalf of,
you may not add another; but you may replace the old one, on explicit
permission from the previous publisher that added the old one.

The author(s) and publisher(s) of the Document do not by this License
give permission to use their names for publicity for or to assert or
imply endorsement of any Modified Version.


\item COMBINING DOCUMENTS

You may combine the Document with other documents released under this
License, under the terms defined in section 4 above for modified
versions, provided that you include in the combination all of the
Invariant Sections of all of the original documents, unmodified, and
list them all as Invariant Sections of your combined work in its
license notice, and that you preserve all their Warranty Disclaimers.

The combined work need only contain one copy of this License, and
multiple identical Invariant Sections may be replaced with a single
copy.  If there are multiple Invariant Sections with the same name but
different contents, make the title of each such section unique by
adding at the end of it, in parentheses, the name of the original
author or publisher of that section if known, or else a unique number.
Make the same adjustment to the section titles in the list of
Invariant Sections in the license notice of the combined work.

In the combination, you must combine any sections Entitled ``History''
in the various original documents, forming one section Entitled
``History''; likewise combine any sections Entitled ``Acknowledgements'',
and any sections Entitled ``Dedications''.  You must delete all sections
Entitled ``Endorsements''.


\item COLLECTIONS OF DOCUMENTS

You may make a collection consisting of the Document and other
documents released under this License, and replace the individual
copies of this License in the various documents with a single copy
that is included in the collection, provided that you follow the rules
of this License for verbatim copying of each of the documents in all
other respects.

You may extract a single document from such a collection, and
distribute it individually under this License, provided you insert a
copy of this License into the extracted document, and follow this
License in all other respects regarding verbatim copying of that
document.


\item AGGREGATION WITH INDEPENDENT WORKS

A compilation of the Document or its derivatives with other separate
and independent documents or works, in or on a volume of a storage or
distribution medium, is called an ``aggregate'' if the copyright
resulting from the compilation is not used to limit the legal rights
of the compilation's users beyond what the individual works permit.
When the Document is included in an aggregate, this License does not
apply to the other works in the aggregate which are not themselves
derivative works of the Document.

If the Cover Text requirement of section 3 is applicable to these
copies of the Document, then if the Document is less than one half of
the entire aggregate, the Document's Cover Texts may be placed on
covers that bracket the Document within the aggregate, or the
electronic equivalent of covers if the Document is in electronic form.
Otherwise they must appear on printed covers that bracket the whole
aggregate.


\item TRANSLATION

Translation is considered a kind of modification, so you may
distribute translations of the Document under the terms of section 4.
Replacing Invariant Sections with translations requires special
permission from their copyright holders, but you may include
translations of some or all Invariant Sections in addition to the
original versions of these Invariant Sections.  You may include a
translation of this License, and all the license notices in the
Document, and any Warranty Disclaimers, provided that you also include
the original English version of this License and the original versions
of those notices and disclaimers.  In case of a disagreement between
the translation and the original version of this License or a notice
or disclaimer, the original version will prevail.

If a section in the Document is Entitled ``Acknowledgements'',
``Dedications'', or ``History'', the requirement (section 4) to Preserve
its Title (section 1) will typically require changing the actual
title.


\item TERMINATION

You may not copy, modify, sublicense, or distribute the Document
except as expressly provided under this License.  Any attempt
otherwise to copy, modify, sublicense, or distribute it is void, and
will automatically terminate your rights under this License.

However, if you cease all violation of this License, then your license
from a particular copyright holder is reinstated (a) provisionally,
unless and until the copyright holder explicitly and finally
terminates your license, and (b) permanently, if the copyright holder
fails to notify you of the violation by some reasonable means prior to
60 days after the cessation.

Moreover, your license from a particular copyright holder is
reinstated permanently if the copyright holder notifies you of the
violation by some reasonable means, this is the first time you have
received notice of violation of this License (for any work) from that
copyright holder, and you cure the violation prior to 30 days after
your receipt of the notice.

Termination of your rights under this section does not terminate the
licenses of parties who have received copies or rights from you under
this License.  If your rights have been terminated and not permanently
reinstated, receipt of a copy of some or all of the same material does
not give you any rights to use it.


\item FUTURE REVISIONS OF THIS LICENSE

The Free Software Foundation may publish new, revised versions of the
GNU Free Documentation License from time to time.  Such new versions
will be similar in spirit to the present version, but may differ in
detail to address new problems or concerns.  See
http://www.gnu.org/copyleft/.

Each version of the License is given a distinguishing version number.
If the Document specifies that a particular numbered version of this
License ``or any later version'' applies to it, you have the option of
following the terms and conditions either of that specified version or
of any later version that has been published (not as a draft) by the
Free Software Foundation.  If the Document does not specify a version
number of this License, you may choose any version ever published (not
as a draft) by the Free Software Foundation.  If the Document
specifies that a proxy can decide which future versions of this
License can be used, that proxy's public statement of acceptance of a
version permanently authorizes you to choose that version for the
Document.

\item RELICENSING

``Massive Multiauthor Collaboration Site'' (or ``MMC Site'') means any
World Wide Web server that publishes copyrightable works and also
provides prominent facilities for anybody to edit those works.  A
public wiki that anybody can edit is an example of such a server.  A
``Massive Multiauthor Collaboration'' (or ``MMC'') contained in the site
means any set of copyrightable works thus published on the MMC site.

``CC-BY-SA'' means the Creative Commons Attribution-Share Alike 3.0
license published by Creative Commons Corporation, a not-for-profit
corporation with a principal place of business in San Francisco,
California, as well as future copyleft versions of that license
published by that same organization.

``Incorporate'' means to publish or republish a Document, in whole or in
part, as part of another Document.

An MMC is ``eligible for relicensing'' if it is licensed under this
License, and if all works that were first published under this License
somewhere other than this MMC, and subsequently incorporated in whole or
in part into the MMC, (1) had no cover texts or invariant sections, and
(2) were thus incorporated prior to November 1, 2008.

The operator of an MMC Site may republish an MMC contained in the site
under CC-BY-SA on the same site at any time before August 1, 2009,
provided the MMC is eligible for relicensing.


ADDENDUM: How to use this License for your documents

To use this License in a document you have written, include a copy of
the License in the document and put the following copyright and
license notices just after the title page:

    Copyright (c)  YEAR  YOUR NAME.
    Permission is granted to copy, distribute and/or modify this document
    under the terms of the GNU Free Documentation License, Version 1.3
    or any later version published by the Free Software Foundation;
    with no Invariant Sections, no Front-Cover Texts, and no Back-Cover Texts.
    A copy of the license is included in the section entitled ``GNU
    Free Documentation License''.

If you have Invariant Sections, Front-Cover Texts and Back-Cover Texts,
replace the ``with \ldots Texts.'' line with this:

    with the Invariant Sections being LIST THEIR TITLES, with the
    Front-Cover Texts being LIST, and with the Back-Cover Texts being LIST.

If you have Invariant Sections without Cover Texts, or some other
combination of the three, merge those two alternatives to suit the
situation.

If your document contains nontrivial examples of program code, we
recommend releasing these examples in parallel under your choice of
free software license, such as the GNU General Public License,
to permit their use in free software.
\end{enumerate}


\printindex

\end{document}
