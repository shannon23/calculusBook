\chapter{Limits}
A sequence is an ordered collection of numbers.  Some examples of sequences:
\begin{itemize}
\item $1,1,1,1,1,1,\ldots$ - a boring sequence
\item $1,2,3,4,5,6,\ldots$ - now we are getting somewhere!
\item $1,\frac{1}{2}, \frac{1}{3},\frac{1}{4},\frac{1}{5},\frac{1}{6},\ldots$ - a very famous sequence, more to come.
\item $2,-2,2,-2,2,-2, \ldots$ - the waffler
\end{itemize}
One of the questions commonly asked about sequences is whether or not the sequence is getting close to some value.  For the first sequence, this is not an interesting question.  The number list starts at 1 and stays there.  We are pretty sure that sequence is ``going to 1''.

The second sequence has a clear pattern - it is getting larger with each term.  If someone tries to convince us that this sequence is going to any particular number, say 5 billion, we can look at term 5,000,000,001 to see that this is not the case.  If this sequence goes somewhere numerically, it goes somewhere big.  We call this big numerical place infinity (denoted $\infty$).  Keep in mind, infinity is not a number, it is an idea of something larger than all the numbers.

\section{Limits}
\begin{mythm} Suppose that $f$ and $g$ are functions such that $f(x)=g(x)$ for all $x$ in some open interval containing $a$ except possibly for $a$, then
\[
\lim_{x\rightarrow a} f(x) = \lim_{x\rightarrow a} g(x)
\]
\end{mythm}

\begin{mythm} Suppose that $f$ and $g$ are functions such that the two limits
\[
\lim_{x\rightarrow a} f(x) = f(a) \textrm{ and } \lim_{x\rightarrow a} g(x) = g(a)
\]
exist.  Suppose that $c$ is a constant and suppose that $n$ is a positive integer.  Then
\begin{align}
&\lim_{x\rightarrow a} c = c\\
&\lim_{x\rightarrow a} x = a\\
&\lim_{x\rightarrow a} c f(x) = c  f(a)\\
&\lim_{x\rightarrow a} (f(x)+g(x)) = f(a)+g(a)\\
&\lim_{x\rightarrow a} (f(x)-g(x)) = f(a)-g(a)\\
&\lim_{x\rightarrow a} f(x) g(x) = f(a) g(a)\\
&\lim_{x\rightarrow a} \frac{f(x)}{g(x)} = \frac{f(a)}{g(a)}, \textrm{ provided }g(a) \ne 0\\
&\lim_{x\rightarrow a} x^n = a^n\\
&\lim_{x\rightarrow a} [f(x)]^n = f(a)^n\\
&\lim_{x\rightarrow a} \sqrt[n]{x} = \sqrt[n]{a}\\
&\lim_{x\rightarrow a} \sqrt[n]{f(x)} = \sqrt[n]{f(a)}, \textrm{ provided } f(a) \ge 0 \\
&\textrm{If } f(x) \le g(x) \textrm{ for all } x \ne a \textrm{ then }\lim_{x\rightarrow a} f(x) \le  \lim_{x\rightarrow a} g(x)
\end{align}
\end{mythm}

\begin{mythm} \textbf{Substitution Theorem}
If $f(x)$ is a polynomial or a rational function, then
\[
\lim_{x\rightarrow a} f(x) = f(a)
\]
assuming $f(a)$ is defined.
\end{mythm}


\subsection{Limit of a function}


\subsection{One-sided limit}
words

\section{Limit of a sequence}
words

\section{Continuity}
\begin{mydef} \textbf{Continuity of a function at a point} \index{continuity}
A function  $f$ is said to be continuous at a point  $p$ if
\begin{description}
\item[a)]  $f(x)$ is defined at  $x=p$, and
\item[b)]
\[
\lim_{x\rightarrow p} f(x)=f(p).
\]
\end{description}
In other words, a function  $f$ is continuous at  $p$ if for every  $\epsilon >0$ there exists a  $\delta >0$ such that
\[
|f(x) - f(p)| < \epsilon \textrm{ whenever } |x - p| < \delta
\]
\end{mydef}



\section{Epsilon-Delta Definition of the Limit}
words
